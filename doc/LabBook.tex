%%%%%%%%%%%%%%%%%%%%%%%%%%%%%%%%%%%%%%%%%
% Daily Laboratory Book
% LaTeX Template
%
% This template has been downloaded from:
% http://www.latextemplates.com
%
% Original author:
% Frank Kuster (http://www.ctan.org/tex-archive/macros/latex/contrib/labbook/)
%
% Important note:
% This template requires the labbook.cls file to be in the same directory as the
% .tex file. The labbook.cls file provides the necessary structure to create the
% lab book.
%
% The \lipsum[#] commands throughout this template generate dummy text
% to fill the template out. These commands should all be removed when 
% writing lab book content.
%
% HOW TO USE THIS TEMPLATE 
% Each day in the lab consists of three main things:
%
% 1. LABDAY: The first thing to put is the \labday{} command with a date in 
% curly brackets, this will make a new page and put the date in big letters 
% at the top.
%
% 2. EXPERIMENT: Next you need to specify what experiment(s) you are 
% working on with an \experiment{} command with the experiment shorthand 
% in the curly brackets. The experiment shorthand is defined in the 
% 'DEFINITION OF EXPERIMENTS' section below, this means you can 
% say \experiment{pcr} and the actual text written to the PDF will be what 
% you set the 'pcr' experiment to be. If the experiment is a one off, you can 
% just write it in the bracket without creating a shorthand. Note: if you don't 
% want to have an experiment, just leave this out and it won't be printed.
%
% 3. CONTENT: Following the experiment is the content, i.e. what progress 
% you made on the experiment that day.
%
%%%%%%%%%%%%%%%%%%%%%%%%%%%%%%%%%%%%%%%%%

%----------------------------------------------------------------------------------------
%	PACKAGES AND OTHER DOCUMENT CONFIGURATIONS
%----------------------------------------------------------------------------------------

\documentclass[idxtotoc,hyperref,openany]{labbook} % 'openany' here removes the gap page between days, erase it to restore this gap; 'oneside' can also be added to remove the shift that odd pages have to the right for easier reading

\usepackage[ 
  backref=page,
  pdfpagelabels=true,
  plainpages=false,
  colorlinks=true,
  bookmarks=true,
  pdfview=FitB]{hyperref} % Required for the hyperlinks within the PDF
  
\usepackage{booktabs} % Required for the top and bottom rules in the table
\usepackage{float} % Required for specifying the exact location of a figure or table
\usepackage{graphicx} % Required for including images
\usepackage{listings}
\lstset{% general command to set parameter(s)
basicstyle=\small, % print whole listing small
keywordstyle=\color{red}\itshape,
% underlined bold black keywords
commentstyle=\color{blue}, % white comments
stringstyle=\ttfamily, % typewriter type for strings
showstringspaces=false,
numbers=left, numberstyle=\tiny, stepnumber=1, numbersep=5pt, %
frame=shadowbox,
rulesepcolor=\color{black},
,columns=fullflexible
} %


\newcommand{\HRule}{\rule{\linewidth}{0.5mm}} % Command to make the lines in the title page
\setlength\parindent{0pt} % Removes all indentation from paragraphs

%----------------------------------------------------------------------------------------
%	DEFINITION OF EXPERIMENTS
%----------------------------------------------------------------------------------------

\newexperiment{Load raw data}{This steps contain the process of cleaning the data}
\newexperiment{Correct class of data}{This steps are for correct the class of each of variable}
%\newexperiment{example3}{This is yet another example experiment}
%\newexperiment{table}{This shows a sample table}
%\newexperiment{shorthand}{Description of the experiment}

%---------------------------------------------------------------------------------------

\begin{document}

%----------------------------------------------------------------------------------------
%	TITLE PAGE
%----------------------------------------------------------------------------------------

\frontmatter % Use Roman numerals for page numbers
\title{
\begin{center}
\HRule \\[0.4cm]
{\Huge \bfseries The Lab note book for recode Network Analysis of Survey \\[0.5cm] \Large Ph.D. of Plant Pathology}\\[0.4cm] % Degree
\HRule \\[1.5cm]
\end{center}
}
\author{\Huge Sith Jaisong \\ \\ \LARGE s.jaisong@irri.org \\[2cm]} % Your name and email address
\date{Beginning 5 February 2015} % Beginning date
\maketitle
\tableofcontents
\mainmatter % Use Arabic numerals for page numbers

%----------------------------------------------------------------------------------------
%	LAB BOOK CONTENTS
%----------------------------------------------------------------------------------------

% Blank template to use for new days:

%\labday{Day, Date Month Year}

%\experiment{}

%Text

%-----------------------------------------

%\experiment{}

%Text

%----------------------------------------------------------------------------------------
\labday{Friday, 6 February 2015}

\experiment{Raw data}

The {\tt{R}} script is to load the raw data from folder.This script start using gdata package \cite{package:gdata}\newline
\lstset{language=R}
The {\tt{R}} script file:
\begin{lstlisting}
######################################
#'title         : 1-raw.R
#'date          : January, 2015
#'purpose       : Load raw data from 
#'                format
#'writed by     : Sith Jaisong (s.jaisong@irri.org)
#'contact       : International Rice Research Institute
#'input         : the xts file                
#'output        : data.frame and RData 
######################################
# The is the script for loading the raw data from local computer
#------Load Library-----
 library(gdata)
#---- Set working directory 
# set your working directory
#wd = '~/Documents/R.github/network.analysis.skep1' 
#setwd(wd)

Filepath <- '~/Google Drive/1.SKEP1/SKEP1survey.xls'
#-----Load raw data (Survey data in SKEP 1)-----

data <- read.xls(Filepath, 
                 sheet = 1, 
                 header = TRUE,
                 stringsAsFactor = FALSE)

#------ Examine the raw data ------

head(data) 
str(data) # check the class of each variable
summary(data)
#---save data to R object ----

save(data, file = "output/1-raw.skep1survey.RData")

\end{lstlisting}

%-----------------------------------------

\experiment{correct class of data} 

The {\tt{R}} script is to load the raw data from folder.This script start using plyr, dplyr, lubridate package \cite{package:plyr, package:dplyr, package:lubridate}\newline


\begin{lstlisting}
####################################################
#'title         : 2-technically_correct
#'date          : January, 2015
#'purpose       :  
#'writed by     : Sith Jaisong (s.jaisong@irri.org)
#'contact       : International Rice Research Institute
#'input         : import excel file from the shared files and delete the 
#'output        : data frame and RData 
#####################################################
# 
library(plyr)
library(dplyr)
library(lubridate)
#-----Load file from output folder-----

load(file = "output/1-raw.skep1survey.RData")


#----- clean define the missing value -----

data[data == "-"] <- NA # replace '-' with NA
data[data == ""] <- NA # replace 'missing data' with NA

#----- to lower variable names ----- 
names(data) <- tolower(names(data))

#----- This step is to seelect the numeric data set -----

data <- transform(data, 
                  phase = as.factor(phase),
                  fno  = as.character(fno),
                  identifier = as.character(identifier),
                  country = as.factor(country),
                  year = as.factor(year),
                  season  = as.character(season),   
                  lat = as.numeric(lat),
                  long = as.numeric(long),      
                  village = as.character(village), 
                  fa = as.numeric(fa),
                  fn = as.character(fn),
                  lfm = as.character(lfm),
                  pc = as.factor(pc),
                  fp = as.character(fp),        
                  cem = as.factor(cem),     
                  ast = as.factor(ast),       
                  nplsqm = as.numeric(nplsqm),
                  ced = dmy(ced),       
                  cedjul = as.numeric(cedjul),
                  hd = dmy(hd),
                  hdjul = as.numeric(hdjul),     
                  ccd = as.character(ccd),
                  cvr = as.character(cvr),
                  vartype = as.character(vartype),
                  varcoded = as.character(varcoded),
                  fym = as.character(fym),
                  fym.coded = as.character(fym.coded),
                  n = as.numeric(n),
                  p = as.numeric(p) ,
                  k = as.numeric(k),
                  mf = as.numeric(mf),        
                  wcp = as.character(wcp),      
                  mu = as.character(mu) ,     
                  iu = as.numeric(iu),     
                  hu = as.numeric(hu),      
                  fu = as.numeric(fu),      
                  cs  = as.factor(cs),      
                  ldg  =  as.numeric(ldg),  
                  yield = as.numeric(yield) ,
                  dscum = as.numeric(dscum),   
                  wecum = as.numeric(wecum),   
                  ntmax = as.numeric(ntmax), 
                  npmax = as.numeric(npmax),    
                  nltmax = as.numeric(nltmax),  
                  nlhmax = as.numeric(nltmax),  
                  waa = as.numeric(waa),      
                  wba = as.numeric(wba) ,   
                  dhx =  as.numeric(dhx),  
                  whx =  as.numeric(whx),     
                  ssx  = as.numeric(ssx),  
                  wma = as.numeric(wma), 
                  lfa = as.numeric(lfa),
                  lma = as.numeric(lma),   
                  rha  = as.numeric(rha) ,
                  thrx = as.numeric(thrx),    
                  pmx = as.numeric(pmx),    
                  defa  = as.numeric(defa) ,
                  bphx = as.numeric(bphx),   
                  wbpx = as.numeric(wbpx),    
                  awx  = as.numeric(awx), 
                  rbx =as.numeric(rbx),   
                  rbbx = as.numeric(rbbx),  
                  glhx  = as.numeric(glhx), 
                  stbx=as.numeric(stbx),    
                  rbpx = as.numeric(rbpx), 
                  hbx= as.numeric(hbx),
                  bbx = as.numeric(bbx),    
                  blba = as.numeric(blba),    
                  lba = as.numeric(lba),    
                  bsa = as.numeric(bsa),    
                  blsa = as.numeric(blsa),  
                  nbsa = as.numeric(nbsa),  
                  rsa  = as.numeric(rsa),   
                  lsa = as.numeric(lsa),    
                  shbx = as.numeric(shbx) ,  
                  shrx = as.numeric(shrx),    
                  srx= as.numeric(srx),    
                  fsmx = as.numeric(fsmx),   
                  nbx =  as.numeric(nbx),   
                  dpx = as.numeric(dpx),    
                  rtdx  = as.numeric(rtdx),  
                  rsdx  = as.numeric(rsdx),
                  gsdx  =as.numeric(gsdx),   
                  rtx = as.numeric(rtx)
) 

save(data, file="output/2-correct.class.skep1survey.RData")

\end{lstlisting}

%\begin{figure}[H] % Example of including images
%\begin{center}
%\includegraphics[width=0.5\linewidth]{example_figure}
%\end{center}
%\caption{Example figure.}
%\label{fig:example_figure}
%\end{figure}

%-----------------------------------------
%\experiment{example3}
%----------------------------------------------------------------------------------------
%\labday{Friday, 26 March 2010}
%\experiment{table}
%\begin{table}[H]
%\begin{tabular}{l l l}
%\toprule
%\textbf{Groups} & \textbf{Treatment X} & \textbf{Treatment Y} \\
%\toprule
%1 & 0.2 & 0.8\\
%2 & 0.17 & 0.7\\
%3 & 0.24 & 0.75\\
%4 & 0.68 & 0.3\\
%\bottomrule
%\end{tabular}
%\caption{The effects of treatments X and Y on the four groups studied.}
%\label{tab:treatments_xy}
%\end{table}
%Table \ref{tab:treatments_xy} shows that groups 1-3 reacted similarly to the two treatments but group 4 showed a reversed reaction.
%----------------------------------------------------------------------------------------
%\labday{Saturday, 27 March 2010}
%\experiment{Bulleted list example} % You don't need to make a \newexperiment if you only plan on referencing it once
%This is a bulleted list:
%\begin{itemize}
%\item Item 1
%\item Item 2
%\item \ldots and so on
%\end{itemize}
%-----------------------------------------
%\experiment{example}
%-----------------------------------------
%\experiment{example2}
%----------------------------------------------------------------------------------------
%	FORMULAE AND MEDIA RECIPES
%----------------------------------------------------------------------------------------
%\labday{} % We don't want a date here so we make the labday blank
%\begin{center}
%\HRule \\[0.4cm]
%{\huge \textbf{Formulae and Media Recipes}}\\[0.4cm] % Heading
%\HRule \\[1.5cm]
%\end{center}
%----------------------------------------------------------------------------------------
%	MEDIA RECIPES
%----------------------------------------------------------------------------------------
%\newpage
%\huge \textbf{Media} \\ \\
%\normalsize \textbf{Media 1}\\
%\begin{table}[H]
%\begin{tabular}{l l l}
%\toprule
%\textbf{Compound} & \textbf{1L} & \textbf{0.5L}\\
%\toprule
%Compound 1 & 10g & 5g\\
%Compound 2 & 20g & 10g\\
%\bottomrule
%\end{tabular}
%\caption{Ingredients in Media 1.}
%\label{tab:med1}
%\end{table}
%-----------------------------------------
%\textbf{Media 2}\\ \\
%Description
%----------------------------------------------------------------------------------------
%	FORMULAE
%----------------------------------------------------------------------------------------
%\newpage
%\huge \textbf{Formulae} \\ \\
%\normalsize \textbf{Formula 1 - Pythagorean theorem}\\ \\
%$a^2 + b^2 = c^2$\\ \\
%-----------------------------------------
%\textbf{Formula X - Description}\\ \\
%Formula
%----------------------------------------------------------------------------------------
\bibliographystyle{plain}
\bibliography{LabBook}
\end{document}